\documentclass[12pt, a4paper]{scrbook}
\usepackage[utf8]{inputenc}
\usepackage{csquotes}
\usepackage[german]{babel}
\usepackage{hyperref}
\usepackage[onehalfspacing]{setspace}
\usepackage{geometry}
\usepackage{color}
\usepackage{listings}
\usepackage{graphicx}
\usepackage{acronym}

%\usepackage[normalem]{ulem}
%\useunder{\uline}{\ul}{}
\usepackage{lstautodedent}
\usepackage{dirtree}

\usepackage[backend=biber,
bibstyle=alphabetic, 
citestyle=authortitle,
natbib=true, 
hyperref=true,
]{biblatex}
\addbibresource{lit.bib} 
\setcounter{secnumdepth}{3}
\setcounter{tocdepth}{3}
\geometry{
left=2.5cm,
right=2.5cm,
top=2.5cm,
bottom=2.5cm,
bindingoffset=5mm,
}
\definecolor{commentgreen}{RGB}{64, 128, 0}
\definecolor{stringred}{RGB}{179, 0, 0}
\definecolor{codebg}{RGB}{249, 248, 238}
\lstdefinestyle{custompython}{
	language=Python,
	backgroundcolor=\color{codebg},
	breaklines=true,
	commentstyle=\color{commentgreen},
	stringstyle=\color{stringred},
	keywordstyle=\color{blue},
	numbers=left,
	showstringspaces=false
}
\pagestyle{empty}
\begin{document}
%\documentclass[titlepage, 12pt]{scrbook}
%\usepackage[utf8]{inputenc}
%\usepackage[german]{babel}
%\usepackage{uarial}
%\usepackage{color}
%\renewcommand{\familydefault}{\sfdefault}
%\usepackage{fancyhdr}
%\usepackage{graphicx}
%\pagestyle{empty}
%\begin{document}
\begin{titlepage}
	\begin{flushleft}
	\begin{figure}
		\hspace*{-0,5cm}
		\includegraphics[scale=0.25]{Bilder/DHBW_logo.jpg} \hspace*{5cm}
		\includegraphics[scale=0.25]{Bilder/Atos_logo.png}
	\end{figure}
	\end{flushleft}
	\vspace*{-0.6cm}
	\begin{center}
	\textcolor{red}{Titel der großen Studienarbeit} \par \vspace*{0,5cm}
	Projektarbeit \par \vspace*{2cm}
	des Studienganges \textcolor{red}{Angewandte Informatik / Betriebliches Informationsmanagement}
	an der Dualen Hochschule Baden-Württemberg Mannheim \par \vspace*{1cm}
	von \par \vspace*{0,5cm}
	\textcolor{red}{Fabian Brandmüller, Maximilian Ludwig, Kevin Wrona} \par \vspace*{1cm}
	\today \par \vspace*{2cm}
	\begin{tabular}{l@{\hspace{3cm}}r}
		Bearbeitungszeitraum & \textcolor{red}{23.09.2019 - 20.04.2019} \\
		Betreuer der DHBW & \textcolor{red}{Eckhard Kruse} \\[1cm] 
	\end{tabular}
	\end{center}
\end{titlepage}
%\end{document}
\setlength{\parindent}{0em} 
\let\cleardoublepage\relax





\section*{Erklärung}
Ich versichere hiermit, dass ich meine Projektarbeit mit dem Thema: ''Titel der großen Studienarbeit'' selbstständig verfasst und keine nderen als die angegeben Quellen und Hilfsmittel benutzt habe.
\newline
\newline
\newline
\newline
---------------------------------------------       ------------------------------------------ \newline
Ort	\hspace{2cm}		Datum\hspace{3,5 cm}				    Unterschrift
\newpage
\section*{Abstract}
\newpage
\begingroup
\renewcommand*{\chapterpagestyle}{empty}
\pagestyle{empty}
\tableofcontents
\listoffigures

\section*{Abkürzungen}

\begin{acronym}[Bash]
% \acro{VCS}{Version Control System}
\end{acronym}
\endgroup
\newpage
\pagestyle{plain}
\setcounter{page}{1}

\chapter{Einleitung}
''Several researchers have stated that facial expression recognition appears to play one of the most important roles in human communication'' 
\footcite[Vgl.][1]{FaceRec}
Dieses Zitat von Katherine B. Leeland gibt einen Einblick in die Relevanz der Emotionserkennung für den Menschen. Fragen zu dieser Thematik stellen sich allerdings nicht erst seit Beginn der Digitalisierung. Bereits Darwin
fragte sich, ob von den Gesichtsausdrücken einer Person nicht auch der Emotionale Zustand abgeleitet werden kann.
\footcite[Vgl.][2]{FaceRec}
Einen solchen Zustand von einem Mitmenschen mittels Software abzulesen ist jedoch nicht nicht leicht zu realisieren. Bereits durch kleine Änderungen in der Mimik werden verschiedene Emotionen ausgedrückt. Zum Beispiel indem eine Person die Lippen zusammen presst und die Augen zusammen kneift bei Wut, oder die Mundwinkel nach unten gezogen werden bei Trauer.
\footcite[Vgl.][249]{HandbookFaceRec}
Durch derartige Ausdrücke können Emotionen wie Wut oder Trauer Ausdruck gewinnen.
Emotionserkennungssoftware gibt es bereits und wird auch in der Wirtschaft eingesetzt. Die Anwendungsgebiete reichen dabei von Jobinterviews, in denen analysiert wird in wie weit die Bewerber für
den jeweiligen Job geeignet sind, 
\footcite[Vgl.][]{mixedArticle}
bis hin zur Automobilindustrie. Dort wird mittels geeigneter Sensorik versucht die Emotion und somit der physiologische Zustand des Autofahrers zu analysieren.
\footcite[Vgl.][Herausforderung]{Frauenhofer}
Diese Daten legen den Grundbaustein für Warnsysteme, welche den Fahrer darauf hinweisen können, dass sein Zustand ungeeignet zum Betrieb eines Kraftfahrzeugs ist. Jedoch werden
solche Einsatzszenarien auch durchaus kontrovers diskutiert. Auf Kritik stößt unter anderem dass die sogenannten ''Basisemotionen'' - z.B. Wut, Trauer, Ekel, Freude, Furcht, Überraschung - , die verwendet werden um den KIs Emotionserkennung
beizubringen, selbst umstritten sind.
\footcite[Vgl.][]{SZ}
Aber auch ethische bedenken werden zunehmend geäußert, vor allem bezüglich der Anwendungsgebiete. Denn je nach Emotion die erkannt werden soll, liegt die Fehlerrate sehr hoch. So hat das
Frauenhofer Institut, welches an Einsatzgebieten von Emotionserkennung in Fahrzeugen arbeitet, festgestellt, dass eine Emotionserkennung je nach Zielemotion eine Vorhersagekraft zwischen 6 und
95\% haben kann.
\footcite[Vgl.][Ergebnis]{Frauenhofer}
Diese negativen Aspekte treffen jedoch nur teilweise auf das hier behandelte Forschungsprojekt zu, wie im Folgenden dargelegt werden soll:
Basierend auf den zuvor genannten Basisemotionen Wut, Frucht, Trauer, Freude, und Ekel soll in dieser Arbeit getestet werden, in wie weit eine technische Vorhersage der Emotionen mittels küünstlicher Intelligenz möglich ist. Dies erfolgt am Anwendungsbeispiel der Pokerface Erkennung. Mittels der hier entworfenen technischen Lösung soll daher getestet werden in wie fern ein Pokerface, das auch als ein emotional neutraler Zustand definiert werden kann, erkannt werden kann.
Diese Forschungsarbeit hat also nicht das Ziel, dass alle oder eine Emotion korrekt vorhergesagt wird, sondern zu ermitteln wann keine Emotion vorliegt.
Das aus dieser Arbeit hervorgehende Prototyp  ist dabei jedoch theoretisch gesehen nicht an das hier verwendete Fallbeispiel des Pokerspielens gebunden. Die Grundidee dieses Prototypen lässt einige hypothetische Einsatzszenarien in der Praxis zu.
 Diese haben eine gewisse Schnittmenge mit denen von ''normaler'' Emotionserkennungssoftware, jedoch gibt es auch einige weitere. Diese  Im Folgenden werden
einige denkbare Szenarien expliziert:
\begin{itemize}
	\item{Polizeiverhöre}
\end{itemize}
Es ist denkbar, dass eine erweiterte Form des entwickelten Prototypen bei Polizeiverhören eingesetzt werden könnte. Hierdurch könnten Beamte die Anwendung eines Pokerfaces durch den Beschuldigten erkennen, welches auf eine Lüge hinweisen könnte. Neben der gebräuchlichen Verwendung eines Lügendetektors wäre der Einsatz der in dieser Arbeit erstellten Lösung eine kostengünstige Variante.
\begin{itemize}
	\item{Gerichtsverhandlungen}
\end{itemize}
Das zweite Einsatzgebiet ist ähnlich zu dem ersten. Bei Gerichtsverhandlungen gelten die gleichen Voraussetzungen wie bei einem Verhör der Polizei. Zwar müssen die Vorgeladenen eine
eidesstattliche Erklärung abgeben nur die Wahrheit zu sagen, jedoch ist zu bezweifeln ob dies auch immer der Fall ist.
Nun soll nicht der Eindruck entstehen dass das hier gebaute Werkzeug ein Lügendetektor ist. Es ist ebenfalls nicht möglich, dass von einem Pokerface immer auf eine Lüge geschlossen werden kann.
Jedoch ist ein Pokerface ein Zeichen dafür, dass sich diese Person ihren emotionalen Zustand nicht anmerken lassen möchte. Und dies wiederum deutet eher daraufhin dass die Person nicht die
Wahrheit sagt oder nur teilweise.
\begin{itemize}
	\item{Pokerspiel}
\end{itemize}
Wie bereits erwähnt ist dieses Einsatzszenario das Fallbeispiel dieser Arbeit. Dies liegt unter anderem daran, dass der erste Begriff der mit dem Wort Pokerface - bzw. einem emotionslosen Gesichtsausdruck -  in Verbindung gebracht wird, das Pokerspiel selber ist. 
Und auch in diesem kann es nützlich sein zu wissen, ob die Kontrahenten ein Pokerface aufsetzen, oder nicht. 
Denkbar wäre es, dass ein Mitspieler zum Beispiel mittels einer Kamera das Gesicht des Gegenübers scannt und analysiert ob ein Pokerface vorliegt oder nicht, und dementsprechend agiert.

\chapter{Anforderungen}
Das nun folgende Kapitel thematisiert die konkrete Aufgabenstellung der Arbeit, so wie eine Anforderungsanalyse mittels MoSCoW Priorisierung. 
\section{Aufgabenstellung}
Das Projekt selber wird an der DHBW in Mannheim durchgeführt und von Prof. Dr. Erckhard Kruse betreut.
Wie eingangs erwähnt soll mittels künstlicher Intelligenz erkannt werden, ob eine Emotion vorliegt oder nicht.
Zur Umsetzung dieser Aufgabe wird eine Bilderkennungssoftware angefertigt, welches ein übermitteltes Bild nach vorhandenen Emotionen analysiert. Sollte keine Emotion durch die Software erkannt werden, wird dem Anwender das Vorhandensein eines Pokerfaces zurück gegeben. Ein konkretes Einsatzgebiet nach Abschluss der Entwicklung ist nicht
vorgesehen, da es sich um ein Forschungsprojekt handelt. Jedoch sind wie bereits in der Einleitung beschrieben einige verschiedene Einsatzmöglichkeiten denkbar, an die das Werkzeug leicht angepasst werden kann.

\section{Usecase}
Wie bereits erwähnt ist der hier behandelte Use Case das Pokerspiel selber. An diesem soll ermittelt werden, in wie weit sich Emotionen vorhersagen lassen können, bzw. das Abhandensein von Emotionen. Dieses Fallbeispiel wurde gewählt, da es unter anderem das naheliegendste ist, so wie ein simples und leicht zu konstruierendes Szenario impliziert. Dabei soll mittels einer Kamera ein Bild von einem Pokerspieler aufgenommen werden, und dann mittels dem hier entworfenen Prototypen verarbeitet und anbalysiert werden. Am Ende soll dann eine Vorhersage der Emotionen getätigt werden die dem Photographen zur Verfügung gestellt wird.


\section{MoSCoW Priorisierung}
Diese Arbeit soll methodisch mit der MoSCoW Priorisierung bearbeitet werden. Diese Art der Priorisierung teilt die zu bearbeitenden Anforderungen in vier Kategorien ein:
\footcite[vgl.][90]{Projektmanagement}
\begin{itemize}
\item Must - Core Anforderungen die unbedingt umgesetzt werden müssen
\item Should - Anforderungen die ebenfalls umgesetzt werden müssen, jedoch um Nachhinein noch durch Change Request verändert werden können.
\item Could - Anforderungen die Nach den Must und Should Anforderungen umgesetzt werden sollen, sofern noch Ressourcen und Zeit vorhanden sind um diese zu bearbeiten
\item Won't - Anforderungen die nicht in diesem Projekt bzw. Release erfolgen, jedoch in einer zukünftigen Version bearbeitet werden sollen. 
\end{itemize}
Im Zuge des Projektes wurden die vorhandenen Anforderungen wie folgt anhand der MoSCoW Priorisierung eingeordnet:
\begin{itemize}
\item Must
\begin{itemize}
\item Emotionen werden nicht zufällig erkannt
\item Es wird zwischen keine Emotion (Pokerface) und Emotion vorhanden unterschieden
\end{itemize}
\item Should
\begin{itemize}
\item Die Wahrscheinlichkeit zur Erkennung der richtigen Emotion muss über 50\% liegen
\item Es können mindestens fünf verschiedene Emotionen erkannt werden
\item Die Erkennung der Emotion darf nicht länger als 30 Sekunden dauern
\item Die Kosten zur Umsetzung der Lösung müssen dem Nutzen gerecht werden?
\end{itemize}
\item Could
\begin{itemize}
\item Bilder können zu Echtzeit analysiert werden
\item Das Trainieren des Modells darf nicht länger als 72 Stunden dauern
\item Eine Oberfläche zur intuitiven Bedienung der Lösung ohne technisches Verständnis muss vorhanden sein
\end{itemize}
\item Won't
\begin{itemize}
\item Neben dem Analysieren von Bildern ist auch die Analyse von Videos möglich
\end{itemize}
\end{itemize}
Dabei sollen die einzelnen Anforderungen entsprechend ihrer Priorität abgearbeitet werden. So kann am Ende der Erfolg der Arbeit deutlich besser eingeordnet werden

\let\cleardoublepage\relax

\chapter{Stand der Technik}
In diesem Abschnitt soll der aktuelle Forschungs- und Entwicklungsstand im Bereich Emotionserkennung thematisiert werden. Jedoch muss dafür erst einmal eine Unterscheidung der Begrifflichkeiten
Emotionserkennung und Gesichtserkennung erfolgen, da beide gebiete Überschneidungen haben, jedoch inhaltlich und von ihren Zielen verschieden sind.

\section{Gesichtserkennung vs. Emotionserkennung}

\subsection{Gesichtserkennung}
Gesichtserkennung ist eine Disziplin der Informatik in der es darum geht Gesichter wieder zu erkennen, und gegebenenfalls verschiedenen Personen zuzuordnen. Dabei lässt sich der Prozess der
Gesichtserkennung in vier Phasen einteilen, Face ''detection'', ''alignment'', ''feature extraction'' und ''matching'', wie anhand von Grafik \ref{fig:Face Recognition} sichtbar ist.
\footcite[Vgl. ][2]{HandbookFaceRec}
\begin{figure}[h]
\includegraphics[width=\linewidth]{Bilder/FaceRecognition.png}
\caption{ Phasen der Gesichtserkennung \newline Quelle: https://alitarhini.files.wordpress.com/2010/12/untitled1.png }
\label{fig:Face Recognition}
\end{figure}
Die Detection Phase ist dafür verantwortlich um zu erkennen ob Gesichter vorhanden sind in einem Bild, oder aber Video.
\footcite[Vgl. ][2]{HandbookFaceRec}
In der darauffolgenden Alignment Phase hingegen wird die Lokalisierung der Gesichter genauer, indem Gesichtskomponenten wie Augen, Augenbrauen, oder die Nase genauer lokalisiert werden. Dabei
wird das Bild oder Video ebenfalls normalisiert, indem z.B. die Bildbeleuchtung angepasst wird.
\footcite[Vgl. ][2]{HandbookFaceRec}
In der Feature extraction hingegen werden die verschiedenen Gesichtskomponenten wie Augen, Nase, Mund, dem Bild oder Video entnommen. Dies ist ein wichtiger Schritt für weitere Prozesse wie
Eye Tracking oder Face Tracking. Alternativ kann sogar eine bestimmte Person anhand der extrahierten Merkmale erkannt werden.
\footcite[Vgl. ][Abstract]{IEEE}
In der letzten Phase, dem Matching, geht es darum die gewonnen Daten mit den in der Datenbank vorhandenen Gesichtern abzugleichen. Wenn eine genügende Übereinstimmung gefunden wurde, wird ein
Match mit einer Person ausgegeben.
\footcite[Vgl. ][3]{HandbookFaceRec}
Die Anwendungsgebiete von Software die Gesichtserkennung ermöglicht ist mannigfaltig. Sie reicht von Applikationen die ein Gerät wie ein Smartphone entsperren, wenn das Gesicht des Besitzers als
Match ausgegeben wurde, bis hin zur Anwendung in Verbrechensbekämpfung. In jedem dieser Szenarien wird dabei der oben beschriebene Ablauf durchgegangen, und abhängig vom zu liefernden Ergebnis
eine Abschlussaktion vorgenommen.

\subsection{Emotionserkennung}
In diesem Unterkapitel nun sollen Emotionen an sich thematisiert werden, da diese maßgeblich sind für das zu entwickelnde Tool. Eine Definition von Emotionserkennung ist per se nicht schwer zu geben. Prinzipiell beschäftigt sich Emotionserkennung mit der Analyse von Gesichtern und den Emotionen die diese Gesichter darstellen. Jedoch ist der Begriff der Emotionen nicht ganz so einfach zu definieren, wie im folgenden erläutert wird: 

%\begin{itemize}
%\item was ist Emotionserkennu
%\item usecase für emotion recognition
%\item Aiusblick und Kontroverse
%\end{itemize}

\subsubsection{Emotionen}

\begin{itemize}
\item Def. von Emotionen
\end{itemize}
 Grundsätzlich gibt es  verschieden Ansätze Emotionen zu
definieren und einzuteilen. Eine Variante ist dabei die eingangs erwähnte, nicht ganz unumstrittene Einteilung in Basisemotionen. Eine gängige Einteilung ist dabei die verschiedenen
Emotionen in acht Bereiche einzuteilen. Diese Einteilung wurden 1984 von Plutchik postuliert und beinhaltet die Emotionskategorien Angst, Wut, Freude, Trauer, Akzeptanz, Ekel, Erwartung und
Überraschung.
\footcite[Vgl. ][3]{FaceRec}
Jedoch ist dies nicht die einzige mögliche Einteilung. Als weiteres Beispiel teilte MacLean die Emotionen in lediglich sechs Kategorien ein, welche da wären: Verlangen, Wut, Angst,
Niedergeschlagenheit, Freude und Zuneigung.
\footcite[Vgl. ][3]{FaceRec}
Wie sich bereits an den beiden Beispielen zeigt, geht die Meinungen der Forscher dabei  stark auseinander, welche und wie viele Emotionen zu den sogenannten ''Basis Emotionen'' gehören. In
dieser Arbeit werden die Emotionen in sechs Kategorien eingeteilt, in Wut, Trauer, Freude, Ekel, Überraschung und Neutral. Diese Einteilung entspricht an sich keiner gängigen Einteilung, jedoch
wurde diese aus den folgenden Gründen gewählt: \newline
Die hier genannten Emotionen lassen sich gut anhand von Bildern erlernen, da diese zum Teil komplementär und somit eindeutig sind. Es ist aber auch einfacher Testdatensätze zu bekommen für ein
freudiges Gesicht, oder ein überraschtes, als ein Gesicht mit dem emotionalen Ausdruck Akzeptanz. Des Weiteren wurde der Ausdruck ''Neutral'' hinzugefügt. Neutral repräsentiert ein emotionsloses
Gesicht, und somit nach Definition einem Pokerface. Zudem sind die gewählten Emotionen häufig bei dem Test Usecase dieser Arbeit anzutreffen, dem Texas Holdem Poker.

\subsubsection{Abgrenzung zur Gesichtserkennung}
Der grundlegende Unterschied zwischen Emotions- und Gesichtserkennung liegt nun darin, dass bei der Emotionserkennung selber nicht die agierende Person im Vordergrund steht, sondern die Aktion die sie ausführt.
Bei der Gesichtserkennung hingegen spielt lediglich die Rolle wer eine Aktion ausführt, und ob es einen Treffer in der Datenbank gibt, oder nicht. Wegen dieser Unterschiede ist auch die technische Realisierung eines Prototypen, vor allem im Bezug auf die Architektur,  durchaus unterschiedlich. Dies ist jedoch ebenso von den unterschiedlichen Anwendungsszenarien der beiden Verfahren bedingt.
 Denn diese sind ebenso verschieden. Während Gesichtserkennung eher in den Bereich IT-Security oder aber Social Media (Snapchat Filter) eingesetzt wird, ist Emotionserkennung eher Informationsgenerierend.
Zum Beispiel kann durch Emotionserkennung Informationen zugänglich werden wie das Befinden eines Individuums ist, ob emotional betroffen ist, oder aber nicht emotional betroffen wirken möchte und ein Pokerface aufsetzt.
Wegen dieser signifikanten Unterschiede kann daher trotz der Gemeinsamkeiten nicht gesagt werden, dass Emotionserkennung eine Unterkategorie von Gesichtserkennung ist.

\section{Emotionserkennung mithilfe von Deep Learning}
In dem nun folgenden Kapitel wird erörtert wie das Ziel des Prototypen dieser Arbeit - das Erkennen eines Pokerfaces - mittels eines neuronalen Netzes umgesetzte werden kann. Dabei wird weniger auf die generellen Eigenschaften von Neuronalen Netzen Bezug genommen, als auf die in dieser Arbeit spezifischen Aspekte. Diese sind vor allem verschieden Ansätze und Möglichkeiten mittels Machine Learning eine Emotionserkennungssoftware zu erstellen.

\subsection{Machine Learning - Frameworks}
Um die gegebene Aufgabenstellung der Erkennung von Emotionen mittels der Analyse eines Gesichtes umsetzen zu können, musste ein entsprechendes Framework Anwendung finden, welches den Anforderungen gerecht wird. Maschinelles Lernen gehört in der heutigen Softwareentwicklung zu den beliebtesten Themen, wodurch dieses schnelle regelmäßige Änderungen und Weiterentwicklungen erfährt. Dementsprechend werden auf dem Markt auch zahllose kostenlose wie auch kostenpflichtige Frameworks angeboten. Um nur einige bekannte aufzuzählen fallen darunter OpenCV, TensorFlow oder Pandas. Möchte man nun das geeignete Framework für das eigene Projekt ausfindig machen, muss man das gegebene Angebot nach einigen Kriterien filtern. Als erstes stellt sich die Frage, was für eine Art von Applikation man umsetzen möchte. Soll das Projekt Texte analysieren oder wie im Falle dieses Projektes die Emotionen aus einem gegebenen Bild? Welche Programmiersprache wird innerhalb des Projektes eingesetzt? Zudem sind Informationen zur Lizenz und dem Support wichtig, sowie insbesondere die Community. 
\newline
Nach entsprechender erster Selektion musste sich schlussendlich zwischen Dlib und Keras entschieden werden, welche für die Umsetzung der Anforderungen dieses Projektes am besten geeignet schienen. Beide Frameworks werden in den folgenden beiden Unterkapitel dem Leser kurz vorgestellt und anschließend wird ein Fazit gezogen, welches der beiden für dieses Projekt am geeignetsten war.

\subsubsection{Dlib}
Dlib ist nach dessen Entwickler Davis King ein modernes C++ Toolkit, welches Machine-Learning Algorithmen und Tools zur Entwicklung komplexer Software enthält um Probleme aus der echten Welt lösen zu können.\footcite[Vgl.]{netguru}
\newline
Dlib selbst wurde in der Programmiersprache C++ entwickelt und kann durch eine Anbindung auch für Python Projekte eingesetzt werden. Die Software-Bibliothek kann unter den Bedingungen der Boost-Lizenz frei genutzt werden und ist durch entsprechende APIs des jeweiligen Betriebssystems Plattformunabhängig. Ein weiterer Vorteil bietet die Unabhängigkeit der Bibliothek von anderen Bibliotheken. %Von Wikipedia (Vlt. andere Quelle suchen)
Dlib ist seit dem Jahr 2002 in Entwicklung und bietet dementsprechend unzählige Features an, welche für verschiedenste Einsatzgebiete verwendet werden können, wie numerische und graphische Modell-Algorithmen und vor allem Gesichtserkennung.\footcite[Vgl.]{netguru}
Eines der größten Vorteile dieser Bibliothek besteht in der ausführlichen Dokumentation für sämtliche Klassen und Funktionen, wie es nicht häufig der Fall ist für vergleichbare Open-Source Projekte.\footcite[Vgl.]{Dlib}

\subsubsection{Keras}
Die Open-Source Bibliothek Keras wurde im Vergleich zu Dlib im Jahre 2015 von dem Google-Programmierer François Chollet in der Programmiersprache Python entwickelt.\footcite[Vgl.]{Keras}


\subsubsection{Unterschiede}


%\begin{table}[]
%\begin{tabular}{llllllllllllllllll}
%\multicolumn{1}{c}{{\ul \textbf{Software}}} &
%  \multicolumn{1}{c}{{\ul \textbf{Creator}}} &
%  \multicolumn{1}{c}{\textbf{\begin{tabular}[c]{@{}c@{}}Initial \\ Release\end{tabular}}} &
%  \multicolumn{1}{c}{\textbf{\begin{tabular}[c]{@{}c@{}}Software \\ Licence\end{tabular}}} &
%  \multicolumn{1}{c}{{\ul \textbf{Open Source}}} &
%  \multicolumn{1}{c}{{\ul \textbf{Platform}}} &
%  \multicolumn{1}{c}{{\ul \textbf{Written in}}} &
%  \multicolumn{1}{c}{{\ul \textbf{Interface}}} &
%  \multicolumn{1}{c}{{\ul \textbf{OpenMP support}}} &
%  \multicolumn{1}{c}{{\ul \textbf{OpenCL support}}} &
%  \multicolumn{1}{c}{\textbf{\begin{tabular}[c]{@{}c@{}}CUDA\\ support\end{tabular}}} &
%  \multicolumn{1}{c}{\textbf{\begin{tabular}[c]{@{}c@{}}Automatic\\ differentiation\end{tabular}}} &
%  \multicolumn{1}{c}{\textbf{\begin{tabular}[c]{@{}c@{}}Has pretrained\\ models\end{tabular}}} &
%  \multicolumn{1}{c}{\textbf{\begin{tabular}[c]{@{}c@{}}Recurrent\\ nets\end{tabular}}} &
%  \multicolumn{1}{c}{\textbf{\begin{tabular}[c]{@{}c@{}}Convolutional \\ nets\end{tabular}}} &
%  \multicolumn{1}{c}{{\ul \textbf{RBM/DBNs}}} &
%  \multicolumn{1}{c}{\textbf{\begin{tabular}[c]{@{}c@{}}Parallel execution \\ (multi node)\end{tabular}}} &
%  \multicolumn{1}{c}{\textbf{\begin{tabular}[c]{@{}c@{}}Actively\\ Developed\end{tabular}}} \\
%Keras &
%  Francois Chollet &
%  2015 &
%  MIT license &
%  Yes &
%  \begin{tabular}[c]{@{}l@{}}Linux, macOS,\\ Windows\end{tabular} &
%  Python &
%  Python, R &
%  \begin{tabular}[c]{@{}l@{}}Only if using \\ Theano as backend\end{tabular} &
%  \begin{tabular}[c]{@{}l@{}}Can use Theano,\\ Tensorflow or \\ PlaidML as backends\end{tabular} &
%  Yes &
%  Yes &
%  Yes &
%  Yes &
%  Yes &
%  No &
%  Yes &
%  Yes \\
%Dlib &
%  Davis King &
%  2002 &
%  Boost Software License &
%  Yes &
%  Cross-Platform &
%  C++ &
%  C++ &
%  Yes &
%  No &
%  Yes &
%  Yes &
%  Yes &
%  No &
%  Yes &
%  Yes &
%  Yes &
%  
%\end{tabular}
%\end{table}

Libraries werden hier expliziert
\subsection{Supervised vs. Unsupervised Learning}
In diesem Abschnitt werden die beiden Ansätze des Supervised bzw. des Unsupervised Learnings evaluiert. Dabei sollen jedoch beide Begriffe nicht noch ausführich beleuchtet werden. Es ist lediglich zu erwähnen, dass Supervised Learning Algorithmen im Gegensatz zu Unsupervised mit Datensätzen arbeiten, die Labels beinhalten, und so einen Datensatz kategorisieren.  Grundlegende Informationen zu den einzelnen Vorgehensweisen  können aus den Büchern "Hands-On Unsupervised Learning with Python" von Giuseppe Bonaccorso und "Applied Supervised Learning with Python" von Benjamin Johnston und Ishita Mathur entnommen werden.
Nun gilt es zu klären, ob sich für die zu Grunde liegende Aufgabe ein Supervised oder UNsupervised Ansatz eher anbietet. Ein Unsupervised Learning Algorithmus würde sich vor allem anbieten, wenn Zusammenhänge zwischen einzelnen Datensätzen gefunden werden sollen, die vielleicht nicht von Anfang an bekannt oder bewusst sind.
\footcite[Vgl. ][21]{Unsupervised}
 Supervised Learning Algoriuthmen hingegen bieten sich vor allem an, wenn es darum geht einen Prozess zu automatisieren oder aus der Wirklichkeit zu replizieren.
\footcite[Vgl. ][4]{Supervised}
Das in dieser Arbeit zu Grunde liegende Problem ist demnach vor allem für Supervised Learning Algorithmen geeignet. Dies liegt zum einem daran, dass jedem Bild eines Menschen eine Basisemotion zugeordnet werden kann. Zum anderem ist auch der verwendete Datensatz mit dem das Modell trainiert und getestet werden soll ebenfalls gelabelt. Deshalb bietet sich dieses Verfahren am meisten an. Es wäre auch hypothetisch denkbar einen Unsupervised Learning Algorithmus zu verwenden, aber dieser Ansatz wäre suboptimal, da er nicht dem eigentlichen Use case dieses ANsatzes entspricht.


\let\cleardoublepage\relax

\subsection{Machine Learning - Frameworks}
Um die gegebene Aufgabenstellung der Erkennung von Emotionen mittels der Analyse eines Gesichtes umsetzen zu können, musste ein entsprechendes Framework Anwendung finden, welches den Anforderungen gerecht wird. Maschinelles Lernen gehört in der heutigen Softwareentwicklung zu den beliebtesten Themen, wodurch dieses schnelle regelmäßige Änderungen und Weiterentwicklungen erfährt. Dementsprechend werden auf dem Markt auch zahllose kostenlose wie auch kostenpflichtige Frameworks angeboten. Um nur einige bekannte aufzuzählen fallen darunter OpenCV, TensorFlow oder Pandas. Möchte man nun das geeignete Framework für das eigene Projekt ausfindig machen, muss man das gegebene Angebot nach einigen Kriterien filtern. Als erstes stellt sich die Frage, was für eine Art von Applikation man umsetzen möchte. Soll das Projekt Texte analysieren oder wie im Falle dieses Projektes die Emotionen aus einem gegebenen Bild? Welche Programmiersprache wird innerhalb des Projektes eingesetzt? Zudem sind Informationen zur Lizenz und dem Support wichtig, sowie insbesondere die Community. 
\newline
Nach entsprechender erster Selektion musste sich schlussendlich zwischen Dlib und Keras entschieden werden, welche für die Umsetzung der Anforderungen dieses Projektes am besten geeignet schienen. Beide Frameworks werden in den folgenden beiden Unterkapitel dem Leser kurz vorgestellt und anschließend wird ein Fazit gezogen, welches der beiden für dieses Projekt am geeignetsten war.

\subsubsection{Dlib}
Dlib ist nach dessen Entwickler Davis King ein modernes C++ Toolkit, welches Machine-Learning Algorithmen und Tools zur Entwicklung komplexer Software enthält um Probleme aus der echten Welt lösen zu können.\footcite[Vgl.]{netguru}
\newline
Dlib selbst wurde in der Programmiersprache C++ entwickelt und kann durch eine Anbindung auch für Python Projekte eingesetzt werden. Die Software-Bibliothek kann unter den Bedingungen der Boost-Lizenz frei genutzt werden und ist durch entsprechende APIs des jeweiligen Betriebssystems Plattformunabhängig. Ein weiterer Vorteil bietet die Unabhängigkeit der Bibliothek von anderen Bibliotheken. %Von Wikipedia (Vlt. andere Quelle suchen)
Dlib ist seit dem Jahr 2002 in Entwicklung und bietet dementsprechend unzählige Features an, welche für verschiedenste Einsatzgebiete verwendet werden können, wie numerische und graphische Modell-Algorithmen und vor allem Gesichtserkennung.\footcite[Vgl.]{netguru}
Eines der größten Vorteile dieser Bibliothek besteht in der ausführlichen Dokumentation für sämtliche Klassen und Funktionen, wie es nicht häufig der Fall ist für vergleichbare Open-Source Projekte.\footcite[Vgl.]{Dlib}

\subsubsection{Keras}
Die Open-Source Bibliothek Keras wurde im Vergleich zu Dlib im Jahre 2015 von dem Google-Programmierer François Chollet in der Programmiersprache Python entwickelt.\footcite[Vgl.]{Keras}


\subsubsection{Unterschiede}


%\begin{table}[]
%\begin{tabular}{llllllllllllllllll}
%\multicolumn{1}{c}{{\ul \textbf{Software}}} &
%  \multicolumn{1}{c}{{\ul \textbf{Creator}}} &
%  \multicolumn{1}{c}{\textbf{\begin{tabular}[c]{@{}c@{}}Initial \\ Release\end{tabular}}} &
%  \multicolumn{1}{c}{\textbf{\begin{tabular}[c]{@{}c@{}}Software \\ Licence\end{tabular}}} &
%  \multicolumn{1}{c}{{\ul \textbf{Open Source}}} &
%  \multicolumn{1}{c}{{\ul \textbf{Platform}}} &
%  \multicolumn{1}{c}{{\ul \textbf{Written in}}} &
%  \multicolumn{1}{c}{{\ul \textbf{Interface}}} &
%  \multicolumn{1}{c}{{\ul \textbf{OpenMP support}}} &
%  \multicolumn{1}{c}{{\ul \textbf{OpenCL support}}} &
%  \multicolumn{1}{c}{\textbf{\begin{tabular}[c]{@{}c@{}}CUDA\\ support\end{tabular}}} &
%  \multicolumn{1}{c}{\textbf{\begin{tabular}[c]{@{}c@{}}Automatic\\ differentiation\end{tabular}}} &
%  \multicolumn{1}{c}{\textbf{\begin{tabular}[c]{@{}c@{}}Has pretrained\\ models\end{tabular}}} &
%  \multicolumn{1}{c}{\textbf{\begin{tabular}[c]{@{}c@{}}Recurrent\\ nets\end{tabular}}} &
%  \multicolumn{1}{c}{\textbf{\begin{tabular}[c]{@{}c@{}}Convolutional \\ nets\end{tabular}}} &
%  \multicolumn{1}{c}{{\ul \textbf{RBM/DBNs}}} &
%  \multicolumn{1}{c}{\textbf{\begin{tabular}[c]{@{}c@{}}Parallel execution \\ (multi node)\end{tabular}}} &
%  \multicolumn{1}{c}{\textbf{\begin{tabular}[c]{@{}c@{}}Actively\\ Developed\end{tabular}}} \\
%Keras &
%  Francois Chollet &
%  2015 &
%  MIT license &
%  Yes &
%  \begin{tabular}[c]{@{}l@{}}Linux, macOS,\\ Windows\end{tabular} &
%  Python &
%  Python, R &
%  \begin{tabular}[c]{@{}l@{}}Only if using \\ Theano as backend\end{tabular} &
%  \begin{tabular}[c]{@{}l@{}}Can use Theano,\\ Tensorflow or \\ PlaidML as backends\end{tabular} &
%  Yes &
%  Yes &
%  Yes &
%  Yes &
%  Yes &
%  No &
%  Yes &
%  Yes \\
%Dlib &
%  Davis King &
%  2002 &
%  Boost Software License &
%  Yes &
%  Cross-Platform &
%  C++ &
%  C++ &
%  Yes &
%  No &
%  Yes &
%  Yes &
%  Yes &
%  No &
%  Yes &
%  Yes &
%  Yes &
%  
%\end{tabular}
%\end{table}

Libraries werden hier expliziert

\subsection{Supervised vs. Unsupervised Learning}
In diesem Abschnitt werden die beiden Ansätze des Supervised bzw. des Unsupervised Learnings evaluiert. Dabei sollen jedoch beide Begriffe nicht noch ausführich beleuchtet werden. Es ist lediglich zu erwähnen, dass Supervised Learning Algorithmen im Gegensatz zu Unsupervised mit Datensätzen arbeiten, die Labels beinhalten, und so einen Datensatz kategorisieren.  Grundlegende Informationen zu den einzelnen Vorgehensweisen  können aus den Büchern "Hands-On Unsupervised Learning with Python" von Giuseppe Bonaccorso und "Applied Supervised Learning with Python" von Benjamin Johnston und Ishita Mathur entnommen werden.
Nun gilt es zu klären, ob sich für die zu Grunde liegende Aufgabe ein Supervised oder UNsupervised Ansatz eher anbietet. Ein Unsupervised Learning Algorithmus würde sich vor allem anbieten, wenn Zusammenhänge zwischen einzelnen Datensätzen gefunden werden sollen, die vielleicht nicht von Anfang an bekannt oder bewusst sind.
\footcite[Vgl. ][21]{Unsupervised}
 Supervised Learning Algoriuthmen hingegen bieten sich vor allem an, wenn es darum geht einen Prozess zu automatisieren oder aus der Wirklichkeit zu replizieren.
\footcite[Vgl. ][4]{Supervised}
Das in dieser Arbeit zu Grunde liegende Problem ist demnach vor allem für Supervised Learning Algorithmen geeignet. Dies liegt zum einem daran, dass jedem Bild eines Menschen eine Basisemotion zugeordnet werden kann. Zum anderem ist auch der verwendete Datensatz mit dem das Modell trainiert und getestet werden soll ebenfalls gelabelt. Deshalb bietet sich dieses Verfahren am meisten an. Es wäre auch hypothetisch denkbar einen Unsupervised Learning Algorithmus zu verwenden, aber dieser Ansatz wäre suboptimal, da er nicht dem eigentlichen Use case dieses ANsatzes entspricht.


\let\cleardoublepage\relax
\chapter{Ergebnis}
In diesem Kapitel wird das Konzept bezüglich der Architektur sowie verwendeter Komponenten und deren Kommunikation dargestellt. Außerdem werden die Aufgrund des erstellten Konzeptes entwickelten Lösungsansätze und Lösungen erläutert. Darin inbegriffen ist vor allem auch das trainierte Modell zum klassifizieren von den vordefinierten Emotionen und das Verifizieren und Testen dieses Modells.

\section{Konzept}
Nachfolgend wird das in der Planungsphase entwickelte Konzept zur Emotionserkennung unter verschiedenen Aspekten erläutert.

\subsection{Architektur}

\subsubsection{Hardware}
Der Einfachheit halber wird als grundlegende Komponente ein handelsüblicher Laptop zum Entwickeln genutzt.  Außerdem wird dabei auf das OpenSource Betriebssystem Ubuntu 18.04.4 LTS in der 64-bit Variante zurückgegriffen. Als zugrunde liegende Ressourcen stehen ein 8 GigaByte Arbeitsspeicher sowie ein Intel Core i5-4210 Quadcore Prozessor mit einer Taktfrequenz von 4 x 2,60 GHz zur Verfügung. Des Weiteren kann die dedizierte Grafikkarte GeForce 820M mit einem Grafikkartenspeicher von 2046 MB verwendet werden. Der Festplattenspeicher von 500 GB kann im Umfang dieser Arbeit vernachlässigt werden, da es im Zusammenhang mit Gesichts- bzw. Emotionserkennung primär darauf ankommt, wie viel Rechenleistung zur Verfügung steht und nicht wie groß die Speicherkapazität des Laufwerks ist. Um das rechenintensive Trainieren des Modells zur Emotionserkennung in akzeptabler Zeit zu garantieren, soll die Rechenleistung des Prozessors und der Grafikkarte vollständig genutzt werden können.

\subsubsection{Programmierumgebung}
Aufgrund der Vorkenntnisse und Relevanz innerhalb des Studiums soll die Programmiersprache Python in aktuellster Version (3.6.9) vorrangig verwendet werden, jedoch kann unter gegeben Umständen auch die Programmiersprache C++ eingesetzt werden, was noch zu prüfen ist. Auf eine bestimmte IDE wie PyCharm, Emacs oder Visual Studio Code wird sich an dieser Stelle nicht festgelegt, da dies jedem Entwickler selbst zu überlassen ist. Diesbezüglich sind nur Einschränkungen aufgrund der gewählten Rechnerarchitektur und der Programmiersprache zu berücksichtigen.\newline
Zur Entwicklung einer Emotionserkennung sollen als grundlegende Komponenten die Programmbibliothek für Bildverarbeitung OpenCV und das Framework Tensorflow bzw. die Deep-Learning-Bibliothek Keras verwendet werden. Da das Einsatzgebiet von OpenCV in der Bildverarbeitung liegt, soll die Bibliothek dazu genutzt werden, den Input so zu verändern, dass dieser vom Modell zum trainieren oder vorhersagen einer oder mehrerer Emotionen verwendet werden kann. Für den wesentlichen Teil der Arbeit, das Entwickeln eines Modells, welches menschliche Emotion anhand eines Bildausschnittes von einem Gesichts erkennen kann, ist die Deep-Learning-Bibliothek Keras zu verwenden.

\subsection{Interaktionskonzept}
Von der Metaebene aus betrachten, steht noch die Planung eines Interaktionskonzeptes aus, welche nun genauer erläutert wird. Die Bildbverarbeitungsbibliothek OpenCV soll dazu verwendet werden, eine grafische Benutzeroberfläche (GUI) in Form eines Fensters zu entwickeln, mit dessen Hilfe dem Nutzer die Möglichkeit geboten wird, entweder Aufnahmen durch die integrierte Webcam selber als Input zu liefern oder dafür ein von ihm gewähltes schon bestehendes Bild auszuwählen. Das vom Nutzer gelieferte Bild soll entsprechend bearbeitet werden, um anschließend eine Emotion zu erkennen und den entstandenen Output an eine durch OpenCV generierte Oberfläche weiterzugeben. In welcher Form der Output dem Nutzer letztendlich dargestellt wird, als Grafik, einfache Textausgabe oder audiovisuell ist noch zu entscheiden.\newline
Zusammenfassend kann man also sagen, dass der Nutzer aufgrund der Übersichtlichkeit und der Einfachheit halber lediglich mit den von OpenCV generierten Oberflächen interagiert und sich zu keinem Zeitpunkt mit der dahinterliegenden Komponente Keras befassen muss.\newline
Es wird sich auch die Option vorbehalten, den Input webbasiert zu liefern und dem Nutzer auch auf gleiche Art und Weise den Output zurückzuliefern. Um die Komplexität des Projektes weiterhin auf die wesentlichen Bestandteile der Emotionserkennung zu beschränken und nicht unnötig zu erhöhen, wird diese Möglichkeit jedoch vorerst nicht weiter berücksichtigt.

\section{Umsetzung der Lösung}
Während der weiteren Ausarbeitung des Konzeptes und der darauf basierenden Umsetzung sind letztendlich zwei verschiedene Lösungen entstanden. Aufgrund von Problemen die innerhalb des Entwicklungsprozesses im Zusammenhang mit dem Hardwarekonzept aufgetreten sind, konnte das Projekt so wie es geplant war nicht zu Ende gebracht werden. Da diese Lösung jedoch als Grundlage für die letztendlich finale Lösung diente, werden nachfolgend beide Umsetzungen genauer erläutert.

\subsection{Stand-Alone Lösung mit Schwerpunkt OpenCV}
In diesem Kapitel wird die Umsetzung der auf dem entwickelten Konzept basierenden Lösung näher erläutert. Es handelt sich dabei um den Stand-Alone Laptop mit installiertem Ubuntu 18.04.4 LTS wobei der Input und der Output auf OpenCV GUIs basiert.

\subsubsection{Input GUI}
Als Einstieg in das Themengebiet Emotionserkennung bzw. Gesichtserkennung bietet sich an, sich zuerst mit den Grundlagen der Bildverarbeitung und den Grundlagen von OpenCV vertraut zu machen. Wie bereits im Konzept beschrieben, soll es dem User möglich sein, mithilfe der Webcam und einer grafischen Oberfläche, welche mit OpenCV programmiert wird, ein Bild als Input zu liefern. Wie man in Abbildung \ref{fig:Input GUI 1} sehen kann, wird im ersten Schritt die grafische Oberfläche erstellt und der Video Stream der Webcam in der Oberfläche angezeigt.
\lstinputlisting[style=custompython, caption=Code für GUI mit Video Stream der Webcam als Output]{Code/InputGUI1.py}
\begin{figure}[h]
\includegraphics[width=\linewidth]{Bilder/InputGUI1.png}
\caption{GUI mit Video Stream der Webcam als Output}
\label{fig:Input GUI 1}
\end{figure}
Nachdem man nun auf den Video Stream der Webcam zugreifen und diesen wiedergeben kann, gilt es den relevanten Bildausschnitt, die sogenannte Region Of Interest (ROI), zu identifizieren. Da wir uns im Umfeld der Gesichts- und Emotionserkennung befinden, sind für uns alle Bereiche relevant, die ein menschliches Gesicht enthalten. Standardmäßig stellt OpenCV einige Modelle zur Objekterkennung zur Verfügung, welche problemlos genutzt werden können. Um nun ein vortrainiertes Modell zur Gesichtserkennung von OpenCV einzubinden, kann man folgende Zeile an den Anfang des Codes schreiben:\newline
\lstinputlisting[style=custompython, numbers=none, linerange=3-3, caption=Einbinden eines vortrainierten OpenCV-Modells zur Gesichtserkennung]{Code/InputGUI2.py}
Mithilfe der Methode \texttt{detectMultiScale} des eingebundenen Modells, können nun die Koordinaten sowie die Höhe und die Breite von jedem in dem Bild gefundenen Gesicht extrahiert werden. Aufgrund dieser Informationen können wir den bisherigen Code nun so erweitern, dass ein Rechteck um jedes gefundene Gesicht im Video Stream der Webcam gezeichnet wird und über die grafische Oberfläche ausgegeben wird. dies sieht dann folgendermaßen aus:\newline
\lstinputlisting[style=custompython, caption=Code für GUI mit Video Stream der Webcam und markierten Gesichtern als Output]{Code/InputGUI2.py}
Wie die durch den Code generiert grafischen Oberfläche aussieht, lässt sich der Abbildung \ref{fig:Input GUI 2} entnehmen.
\begin{figure}[h]
\includegraphics[width=\linewidth]{Bilder/InputGUI2.png}
\caption{GUI mit Video Stream der Webcam und markierten Gesichtern als Output}
\label{fig:Input GUI 2}
\end{figure}
Nun lässt sich noch darüber diskutieren, ob es weitere besonders zu berücksichtigende ROIs gibt. Im Zusammenhang mit Emotionen können unter anderem die Augen und der Mund eine besondere Rolle spielen, sodass nachfolgend einmal beispielhaft das Hinzufügen eines weiteren Modells zur Erkennung der Augen in jedem bereits entdeckten Gesicht:
\lstinputlisting[style=custompython, numbers=none, linerange=4-4, caption=Einbinden eines vortrainierten OpenCV-Modells zur Augenerkennung]{Code/InputGUI3.py}
Um nun alle Augen in einem Bild zu erkennen, kann nun wieder die Methode \texttt{detectMultiScale} des Modells genutzt werden. Dem nachfolgenden Codeschnipsel kann man nun entnehmen, wie um alle Augen, die sich in dem Bildausschnitt befinden in dem auch ein Gesicht erkannt wurde, ein grünes Rechteck gezeichnet wird.
\lstinputlisting[style=custompython, numbers=none, linerange=9-18, showspaces=false, autodedent, caption=Zeichnen von Rechtecken um alle erkannten Augen und Gesichter]{Code/InputGUI3.py}
Somit sind wir nun in der Lage, alle relevanten Bereiche zu identifizieren und zu markieren. Um diese relevanten Regionen nun als input zum Vorhersagen einer Emotion für ein Modell nutzen zu können, müssen die entsprechenden Bereiche jedoch ersteinmal abgespeichert werden. Um eine Vergleichbarkeit der Bilder untereinander und des zu testenden Bildes zu den trainierten Bildern zu gewährleisten, sollten diese alle das gleiche Format besitzen. Wie bereits in der Theorie erwähnt, ist es außerdem sinnvoll, entsprechende Ausschnitte nicht im Farbmodus abzuspeichern, sondern lediglich als Grayscale-Grafik. Erweitert man den bereits entstandenen Code erneut um die erleuterten Aspekte, sieht dies so aus:
\lstinputlisting[style=custompython, caption=Code für GUI mit Video Stream der Webcam und markierten Gesichtern und Augen als Output]{Code/InputGUI3.py}
Das daraus resultierende Ergebnis kann man der Abbildung \ref{fig:Input GUI 3} entnehmen.
\begin{figure}[h]
\includegraphics[width=\linewidth]{Bilder/InputGUI3.png}
\caption{GUI mit Video Stream der Webcam und markierten Gesichtern und Augen als Output}
\label{fig:Input GUI 3}
\end{figure}
Somit ist es dem Nutzer möglich, schnell und unkompliziert Bilder mit der Webcam zu machen und dort enthaltene Gesichter so zu speichern, dass sie als Input zum Vorhersagen einer Emotion genutzt werden können.

\subsubsection{Dataset}
Zum Trainieren eines Modells zur Emotionserkennung, wird das sogenannte Cohn-Kanade Dataset zur Analyse von Emotionen verwendet.\footcite[Vgl.][]{CK} In dem verwendeten Dataset sind ca. 10000 Grayscale-Bilder enthalten, welche die in \ref{tab:ckemotions} zu sehenden Emotionen:
\begin{table}[h]
\centering
\begin{tabular}[t]{l|c}
Emotion & N \\
\hline
Anger & 45 \\
Contempt & 18 \\
Disgust & 59 \\
Fear & 25 \\
Happy & 69 \\
Sadness & 28 \\
Surprise & 83 \\
\hline
\end{tabular}
\caption{Emotionen mit jeweiliger Anzahl an Bildern}
\label{tab:ckemotions}
\end{table}
Es fällt auf, dass sich die dargestellte Anzahl von ca. 325 nutzbaren Bilder stark von der bereits erwähnten Anzahl von 10000 Bildern unterscheidet. Dies liegt daran, dass das Dataset ursprünglich für die Analyse von Emotionen und nicht direkt für die Emotionserkennung entwickelt wurde. Was das kann genau hast, wird durch die Abbildung \ref{fig:Disgust} verdeutlicht.
\begin{figure}[h]
	\begin{minipage}[b]{.2\linewidth} % [b] => Ausrichtung an \caption
		\includegraphics[width=\linewidth]{Bilder/Disgust1.png}
	\end{minipage}
	\hspace{.025\linewidth}% Abstand zwischen Bilder
	\begin{minipage}[b]{.2\linewidth} % [b] => Ausrichtung an \caption
		\includegraphics[width=\linewidth]{Bilder/Disgust2.png}
	\end{minipage}
	\hspace{.025\linewidth}% Abstand zwischen Bilder
	\begin{minipage}[b]{.2\linewidth} % [b] => Ausrichtung an \caption
		\includegraphics[width=\linewidth]{Bilder/Disgust3.png}
	\end{minipage}
	\hspace{.025\linewidth}% Abstand zwischen Bilder
	\begin{minipage}[b]{.2\linewidth} % [b] => Ausrichtung an \caption
		\includegraphics[width=\linewidth]{Bilder/Disgust4.png}
	\end{minipage}
	\newline
	\begin{minipage}[b]{.2\linewidth} % [b] => Ausrichtung an \caption
		\includegraphics[width=\linewidth]{Bilder/Disgust5.png}
	\end{minipage}
	\hspace{.025\linewidth}% Abstand zwischen Bilder
	\begin{minipage}[b]{.2\linewidth} % [b] => Ausrichtung an \caption
		\includegraphics[width=\linewidth]{Bilder/Disgust6.png}
	\end{minipage}
	\hspace{.025\linewidth}% Abstand zwischen Bilder
	\begin{minipage}[b]{.2\linewidth} % [b] => Ausrichtung an \caption
		\includegraphics[width=\linewidth]{Bilder/Disgust7.png}
	\end{minipage}
	\hspace{.025\linewidth}% Abstand zwischen Bilder
	\begin{minipage}[b]{.2\linewidth} % [b] => Ausrichtung an \caption
		\includegraphics[width=\linewidth]{Bilder/Disgust8.png}
	\end{minipage} 
	\newline
	\begin{minipage}[b]{.2\linewidth} % [b] => Ausrichtung an \caption
		\includegraphics[width=\linewidth]{Bilder/Disgust9.png}
	\end{minipage}
	\hspace{.025\linewidth}% Abstand zwischen Bilder
	\begin{minipage}[b]{.2\linewidth} % [b] => Ausrichtung an \caption
		\includegraphics[width=\linewidth]{Bilder/Disgust10.png}
	\end{minipage}
	\hspace{.025\linewidth}% Abstand zwischen Bilder
	\begin{minipage}[b]{.2\linewidth} % [b] => Ausrichtung an \caption
		\includegraphics[width=\linewidth]{Bilder/Disgust11.png}
	\end{minipage}
	\caption[t]{Emotionsverlauf von Neutral zu Disgust}
	\label{fig:Disgust}
\end{figure}

Jeder Datensatz beinhaltet mehrere Bilder, die den Verlauf einer Emotion von Neutral bis hin zur Emotion darstellen. Da dieser Verlauf für die Emotionserkennung allenfalls für die Klassifizierung innerhalb der jeweiligen Emotion relevant wäre und somit im Umfang dieser Arbeit nicht betrachtet wird, wird nur das jeweils letzte Bild dieses Verlaufs, also die vollständig ausgeprägte Emotion, zum Trainieren des Modells genutzt. Somit ergibt sich der Unterschied von der Anzahl der jeweils letzten Bilder der Verläufe, ca. 325, und der Anzahl der Gesamtbilder aller Verläufe, ca. 10000.
Zur Verzeichnisstruktur des Datasets kann sagen, dass es grundsätzlich zwei verschiedene Ordner gibt. In einem Ordner befinden sich alle Bilder, also die Verläufe der Emotionen und in dem anderen befinden sich mit der gleichen Struktur die dazugehörigen Emotionen. Stellt man die Struktur der beiden Ordner beispielhaft als Verzeichnisbaum dar, sieht dies folgendermaßen aus:
\setlength{\DTbaselineskip}{20pt}
\DTsetlength{1em}{1.5em}{0.2em}{1pt}{4pt}
\renewcommand*\DTstylecomment{\rmfamily\color{commentgreen}\textsc}
\renewcommand*\DTstyle{\ttfamily\textcolor{red}}
\begin{figure}
\begin{minipage}[b]{.5\linewidth}
\dirtree{%
.1 images.
.2 S005\DTcomment{Person 1}.
.3 001.
.2 S010\DTcomment{Person 2}.
.3 001.
.3 002.
.3 003.
.2 S011\DTcomment{Person 3}.
.3 001.
.3 002.
.2 S014\DTcomment{Person 4}.
.3 001.
.3 002.
.3 003.
.3 004.
.3 005.
.2 \ldots.
}
\end{minipage}
\begin{minipage}[b]{.5\linewidth}
\dirtree{%
.1 emotions.
.2 S005\DTcomment{Person 1}.
.3 001.
.2 S010\DTcomment{Person 2}.
.3 001.
.3 002.
.3 003.
.2 S011\DTcomment{Person 3}.
.3 001.
.3 002.
.2 S014\DTcomment{Person 4}.
.3 001.
.3 002.
.3 003.
.3 004.
.3 005.
.2 \ldots.
}
\end{minipage}
\caption{fig:Test}
\end{figure}
Wie man den Verzeichnisbäumen entnehmen kann, ist die Struktur in beiden Ordnern identisch. In der ersten Ebene befinden sich Ordner jeweils mit 'S' aufgeteilt nach Personen. Innerhalb jeder dieser Personenordner gibt es einen oder mehrere weitere Ordner. Diese Ordner auf der 2. Ebene sind dreistellig aufsteigend numeriert und beinhalten jeweils einen Emotionsverlauf. Im Sachzusammenhang bedeutet dies, dass jede Person eine oder mehrere verschiedene Emotionsverläufe darstellt. Die Syntax der Dateinamen in den aufsteigend numerierten Ordner, die letztendlich den Emotionsverlauf beinhalten, ist \texttt{<Personenordner>\_<Emotionsverlaufsordner>\_<Pos. im Verlauf>.png} wobei die Position im Verlauf 8-stellig ist. Nimmt man die in Abbildung \ref{fig:Disgust} als Grundlage, könnten die einelnen Dateinamen zu den Bildern wie im Verzeichnisbaum \ref{fig:Tree Emotionsverlauf} aussehen.
\begin{figure}
\dirtree{%
.1 images.
.2 S005\DTcomment{Person 1}.
.3 001\DTcomment{Emotionsverlauf 1}.
.4 S005\_001\_00000001.png.
.4 S005\_001\_00000002.png.
.4 S005\_001\_00000003.png.
.4 S005\_001\_00000004.png.
.4 S005\_001\_00000005.png.
.4 S005\_001\_00000006.png.
.4 S005\_001\_00000007.png.
.4 S005\_001\_00000008.png.
.4 S005\_001\_00000009.png.
.4 S005\_001\_00000010.png.
.4 S005\_001\_00000011.png.
.2 \ldots.
}
\caption{Struktur Emotionsverlauf}
\label{fig:Tree Emotionsverlauf}
\end{figure}
Aus der dargestellten Struktur des Ordners in dem sich sämtliche Bilder befinden und den jeweiligen Dateinamen lassen sich keinerlei Rückschlüsse auf die dargestellten Emotionen ziehen. Um herauszufinden welche Emotion durch welchen Emotionsverlauf dargestellt wird, muss man in das entsprechende Verzeichnis in dem Ordner mit den gelabelten Emotionen wechseln. Dort befindet sich dann eine Textdatei, die den selben Dateinamen hat, wie das letzte Bild aus dem dazugehörigen Emotionsverlauf. In dieser Datei befindet sich dann eine Fließkommazahl, die für die jeweilige Emotion in Tabelle \ref{tab:ckemotions} steht. Um nun herauszufinden welche Emotion durch den in \ref{fig:Tree Emotionsverlauf} dargestellt wird, schaut man in das Verzeichnis \texttt{emotions/S005/001/}, in dem sich die Datei \texttt{S005\_001\_00000011.txt} befindet. In dieser Datei würde dann \texttt{3.0000000e+00} stehen, was sich gemäß der Tabelle mit \texttt{Disgust} gleichsetzen lässt.\newline
Um später ein vernüftiges superviesed learning auf Grundlage des Dataset durchzuführen, wird eine eigene Verzeichnisstruktur erstellt, die jedes letzte Bild eines Emotionsverlaufs sortiert nach Emotion beinhaltet. Der folgenden Code automatisiert das überarbeiten und sortieren des Datasets und erweitert dieses um die Emotion \texttt{Neutral}, da jedes erste Bild eines Emotionsverlaufs als neutrale Ausgangslage dient.
\lstinputlisting[style=custompython, caption=Dataset sortieren und überarbeiten]{Code/OrganizingTheDataset.py}
Das überarbeitete und sortierte Dataset, welches nun als neue Grundlage für die Stand-Alone Lösung dient, sieht dann so aus:
\begin{figure}
	\dirtree{%
	.1 sorted\_set.
	.2 anger.
	.3 1.png.
	.3 2.png.
	.3 \ldots.
	.2 contempt.
	.3 1.png.
	.3 2.png.
	.3 \ldots.
	.2 disgust.
	.3 1.png.
	.3 2.png.
	.3 \ldots.
	.2 fear.
	.3 1.png.
	.3 2.png.
	.3 \ldots.
	.2 happy.
	.3 1.png.
	.3 2.png.
	.3 \ldots.
	.2 neutral.
	.3 1.png.
	.3 2.png.
	.3 \ldots.
	.2 sadness.
	.3 1.png.
	.3 2.png.
	.3 \ldots.
	.2 surprise.
	.3 1.png.
	.3 2.png.
	.3 \ldots.
	}
\caption{Struktur überarbeitetes Dataset}
\label{fig:Tree Sorted Set}
\end{figure}

\subsubsection{Training}
OpenCV stellt verschiedene Klassifizierer zur Verfügung, jedoch wird sich in diesem Abschnitt immer auf das Supervised Learning des FisherFaceRecognizer bezogen, welcher mit dem im vorherigen Schritt präparierten Dataset trainiert wird. Da das Dataset schon sortiert und vorbereitet wurde, ist es nun sehr leicht eine Methode zu schreiben, die ein Array mit Trainingsdaten und ein Array mit den zugehörigen Labels liefert. Aufgrund der Verzeichnisstruktur des Datasets kann man über alle Emotionsordner iterieren und alle Grayscale-Bilder in das Array mit den Trainingsdaten einlesen und die jeweilige Emotion in das Array mit den Labels einlesen. Dabei ist jedoch zu beachten, dass die Labels Integer-Werte aufsteigend mit 0 beginnend sein müssen. Das heißt, die Ordnernamen, welche die Emotionen als Text sind, müssen in Integer-Werte konvertiert werden, indem man z.B. den jeweiligen Index in dem Emotions-Array \lstinputlisting[style=custompython, numbers=none, linerange=5-5, caption=Emotions-Array]{Code/ClassifierHandler.py} nimmt. Setzt man dies um, entsteht folgender Code:
\lstinputlisting[style=custompython, numbers=none, linerange=8-17, caption=Trainingsdaten als Arrays extrahieren]{Code/ClassifierHandler.py}
Da wir nun die erforderlichen Arrays mit Trainingsdaten und Labeln erzeugen können, beschäftigen wir uns als nächstes mit dem zu trainierenden Klassifizierer, dem sogenannten FisherFaceRecognizer. Im folgenden werden zwei Methoden dargestellt, mit denen dieser erzeugt und trainiert werden kann.
\lstinputlisting[style=custompython, numbers=none, linerange=19-24, caption=Methoden zum Erstellen und Trainieren eines FisherFaceRecognizer]{Code/ClassifierHandler.py}
Da nun alle erforderlichen Einzelschritte definiert sind, können wir diese nun zusammenfügen. \lstinputlisting[style=custompython, numbers=none, linerange=55-57, caption=Erstellen und Trainieren eines FisherFaceRecognizer]{Code/ClassifierHandler.py}
Wie dem Code zu entnehmen ist, stellt die \texttt{FisherFaceRecognizer}-Klasse die Methoden \texttt{create()} und \texttt{train(InputArray data, InputArray labels)} zur Verfügung, sodass der Großteil der Komplexität des gesamten Trainingsprozesses bei dem Vorbereiten des Datasets liegt.

\subsubsection{Testing}
Das Testen des im vorherigen Schritt trainierten Modells kann aufgrund der Vorarbeit ebenfalls sehr schnell implementiert werden. Aufgrund der wie sich später herausstellte sehr geringen Datengrundlage und der Should-Anforderung gemäß MoSCoW Priorisierung, dass die Wahrscheinlichkeit zu Erkennung der richtigen Emotion über 50\% liegen soll, wird anstatt der üblichen Aufteilung des Datasets in Trainings- und Testdaten das komplette Dataset als Trainingsdaten genutzt. Daraus ergibt sich dann, dass ein anderes Verfahren zum Testen des Klassifiziers gefunden werden muss. Somit wird zum Testen des Modells eine grafische Oberfläche erstellt, die im speziellen als Echtzeitanalyse der Emotionen fungiert. Hierzu kann ein Großteil des Codes genutzt werden, der als Vorarbeit im Bereich der Input GUI implementiert wurde. Der Ablauf eines konkreten Tests ist dann das Starten der Oberfläche, welche den Video Stream der Webcam ausgibt und gleichzeitig jedes Einzelbild vom Modell analysieren lässt. Dies geschieht wie ebenfalls bereits beschrieben durch das Ausschneiden des Gesichtes und das anschließende konvertieren in das Grayscaleformat sowie daraufhin das wesentliche Vorhersagen einer Emotion. Das Gesicht wird dann mit einem Rechteck markiert und darüber wird die vorhergesagte Emotion ausgegeben. Verändert man den Code der Input GUI nur geringfügig und erweitert ihn um das Vorhersagen der Emotion und das Schreiben der vorhergesagten Emotion auf den Output Video Stream, dann erhält man folgendes Ergebnis:
\lstinputlisting[style=custompython, numbers=none, linerange=26-50, caption=Klassifizierer mit der Input GUI testen]{Code/ClassifierHandler.py}

\subsubsection{Optimierung der Lösung}

\subsection{Webbasierte Lösung}

\subsubsection{Dataset}

\subsubsection{Modell}
Nachfolgend wird das für die Emotionserkennung erstellte Modell mit den jeweils einzelnen Layern näher erläutert und \ref{fig:Model Summary}
\begin{figure}[h]
\includegraphics[viewport=0 1177 300 1844]{Bilder/ModelSummary.png}
\end{figure}
\begin{figure}[h]
\includegraphics[viewport=-14 150 300 1030]{Bilder/ModelSummary.png}
\end{figure}
\begin{figure}[h]
\includegraphics[viewport=0 0 455 510]{Bilder/ModelSummary.png}
\caption{ Model Summary with several Layers and I/O Shapes }
\label{fig:Model Summary}
\end{figure}

\subsubsection{Trainieren des Modells}

\subsubsection{Testen des Modells}

\subsubsection{Webserver}

\subsubsection{Jupyter Notebook}

\subsubsection{Verzeichnisstruktur}

\let\cleardoublepage\relax
\chapter{Diskussion}
Das nunmehr letzte Kapitel soll sich mit der kurzen Zusammenfassung der Ergebnisse des letztens Teils und deren Bewertung widmen. 
%Die Ergebnisse aus dem letzten Kapitel noch einmal zusammenfassen?
Des Weiteren sollen die angewandten Methoden reflektiert werden,
offene Fragen beantwortet und auch weitere Punkte aufgezeigt werden die verbessert oder noch implementiert werden können. Dazu soll zunächst die Ergebnisse kurz zusammengefasst werden.

\section{Reflexion der Ergebnisse}

\subsection{Alternativen}

\section{Reflexion Vorgehen}
Mehr darauf eingehen dass das Kontrovers ist und auch die Basisemotionen kontrovers sind --  aber keine andere Möglichkeit vorhanden 

\section{Reflexion der Literatur}
Bezüglich der Literatur ergeben sich nun einige Schwierigkeiten. Dies liegt unter anderem daran, dass das generelle Thema der Gesichts und Emotionserkennung immer noch vor allem aus
psychologischer Sicht in der Literatur behandelt wurde. Zwar gibt es Fachbücher auch aus informationstechnischer Sicht, welche ebenfalls in dieser Arbeit verwendet wurden.

\section{Offene Implikationen}

\chapter{Ausblick}

\section{Alternative Ansätze zur Umsetzung von Emotionserkennung}
In diesem Abschnitt nun werden verschiedene alternative Ansätze dargestellt und expliziert, die dazu verwendet werden können um Emotionen zu erkennen.
Dieses Unterkapitel beschäftigt sich mit alternativen Ansätzen zu den bereits explizierten Basisemotionen. Diese sind wie bereits erwähnt umstritten, was die Frage zulässt warum diese überhaupt
verwendet werden sollten. Ein weiterer kreativer Ansatz zur Erkennung von Emotionen wäre die Analyse der Stimmlage.
Dieser Ansatz beruft sich darauf, dass das Sprachzentrum eines Menschen einer der wichtigsten Aspekte der Kommunikation und somit auch der Preisgabe von Informationen über den emotionalen Zustand eines Individuums ist.
\footcite[Vgl. ][Abstract]{EmotionInSpeech}
Dieser Ansatz ist jedoch nicht zielführend, da hier hauptsächlich die Stimme analysiert wird. Von einer Stimme kann nun auf eine Emotion geschlossen werden. Für den Usecase ist dieser Ansatz allerdings ungeeignet, aus folgenden Gründen:\newline
Es kann möglich sein eine Emotion anhand der Sprache zu erkennen. Das Äquivalent eines Pokerfaces wäre dementsprechend eine neutrale Stimmlage, welche keine Emotionen suggeriert. Nun kann aber keine Aussage getroffen werden aus welchen Gründen eine Person neutral spricht. Es könnte von einem Pokerface stammen, oder einer monotonen Sprechweise, oder einen gelangweilten Gemütszustand. Dies ist nicht eindeutig identifizierbar. Gleiches könnte nicht über ein neutrales Gesicht gesagt werden, da dies gemeinhin als Pokerface bezeichnet wird. %Spricht das nicht auch gegen unser Vorgehen?
Ein weitere Ansatz wäre die Analyse der derzeit vernommenen Musik. Diese kann einem bestimmten Gemütszustand zugesprochen werden, welches auf eine aktuelle Emotion übertragbar ist.
\footcite[Vgl.][1]{MusicEmotion}
Ziel dieses Forschungszweiges ist es daher die hinter Liedern oder Klängen stehenden Emotionen zu ermitteln und diese entsprechend zu kategorisieren.
Dieser Ansatz erscheint zunächst durchaus interessant, hat jedoch genauso Nachteile wie die Analyse von Emotionen anhand von Bildern die Basisemotionen zeigen. %Spricht wieder gegen unser Vorgehen
Dieser liegt hier unter anderem in der Genauigkeit der Analysen. So z.B. lieferte ein Testprojekt an der Russischen HSE (Higher School of Economics) das Ergebnis von einer maximalen Genauigkeit von 71\%.
\footcite[Vgl. ][Abstract]{EmotionInSound}
In dem Versuchsaufbau wurden Spektrogramme von Klangfragmenten ausgewertet und versucht mittels Neuronalen Netzen eine Klassifikation der hinter dem Klang liegenden Emotion zu erreichen.
\footcite[Vgl. ][Abstract]{EmotionInSound}
Der generelle Ansatz anhand von Musik die Emotion eines Individuums abzulesen ist zwar praktikabel und von dem Versuchsaufbau auch vergleichbar zu dem Ansatz bereits gelabelte Bilder zu verwenden. Jedoch lässt sich auf diese Weise aus zwei Gründen nicht die eigentliche Zielaufgabenstellung ableiten, das Erkennen eines Pokerfaces. Zum einen handelt es sich in dieser Arbeit um eine visuelle Problemstellung, in welcher das Erkennen des Gemütszustandes anhand des Gesichtsausdruckes erkannt werden soll, also einem vorhandenen bzw. nicht vorhandenen Pokerface. Zum anderen würde die Analyse von Musik einen Rückschluss auf den allgemeinen Gemütszustand des Betroffenen folgern und nicht eine kurzzeitige Stimmungsschwankung aufgrund beispielsweise eines schlechten Blattes, wie in diesem Usecase.

\let\cleardoublepage\relax
\newpage

\printbibheading
\printbibliography[type=book,heading=subbibliography,title={Literaturquellen}]
\pagestyle{empty}
\printbibliography[type=misc,heading=subbibliography,title={Sonstige Quellen}]
\pagestyle{empty}
\newpage
\pagestyle{empty}

\end{document}
